\documentclass[a4paper,11pt]{article}


\usepackage[latin1]{inputenc}

\usepackage{amsmath}


\newcommand{\nnsection}[1]{
\section*{#1}
\addcontentsline{toc}{section}{#1}
}

\begin{document}

\begin{center}
\vspace{20pt}
\textbf{\large Higher Order Programming}\\
\vspace{10pt}
\textbf{Johan Montelius}\\
\vspace{10pt}
\today{}
\end{center}

\nnsection{Sorting warm up}

As a warm up we can start by implementing a general purpose sort
function using the the merge sort implemented in the recursion
assignment. 

Make a copy of {\tt Sort.hs} and call it {\tt Sortg.hs} for {\em sort
  generic}. Now add an extra argument for the ordering function and
adapt the type declarations accordingly. 

Now we need something to sort and why not a list of persons where we
wan to sort them by age. First define a new data type that will
represent a person by name, gender and age. Then define a function
that determines of one person is younger than another. When you have
all the pieces of the puzzle you can sort a list of persons with out
altering the sorting functions.


\nnsection{Working with map, fold and filter}

The next task is to work with map, filter, fold and related higher
order functions that operate on lists. Implement the following
functions but use only the higher order functions for the recursion. 

\begin{itemize}
\item Given a list of integers, generate a list of all even elements
  raised by the power of two.
\item Given a lists of lists of integers, calculate the sum of the
  length of all lists containing a $5$.
\item Given a list of persons, generate a list of all persons older
  than $35$.
\item Given a list of persons, calculate the average age of all
  persons named ``Adam''.
\item Parse a list of integers representing a decimal number and
  evaluate the resulting integer ({\em [1,3,2] -\textgreater 132})
\end{itemize}

\nnsection{The difference between foldl and foldr}

One way to show the difference between {\em foldl} and {\em foldr} is
to construct a list as the result. First take a list {\em [1..10]} and
calculate the sum using both {\em foldl} and {\em foldr}. Both
versions will look very similar and the result is of course the same.

\begin{itemize}
\item The function {\em right} will used {\em foldr} and the operation {\em (+)}
\item The function {\em left} will used {\em foldl} and the operation {\em (+)}
\end{itemize}

Now we replace the {\em (+)} operation with something that shows in
what order things are done.

\begin{itemize}
\item The function {\em right} will used {\em radd} that takes an
  integer and a string (the accumulated value) and produces a string
  ``( n + \textless ack\textgreater )'' where {\em \textless ack\textgreater} is the accumulated value.
\item The function {\em left} will used {\em ladd} that takes an
a string (the accumulated value) and an integer and produces a string
  ``( \textless ack\textgreater + n)'' where {\em \textless ack\textgreater} is the accumulated value.
\end{itemize}

Use the predefined function {\em show} to convert an integer to a
string and the operator {\em (++)} to append strings together. It
should be quite clear what the two functions do.

\nnsection{Parsing with closure} 

One interesting usage of higher order is to encapsulate a computation
in a so called {\em closure}. We will implement a simple parser that
takes a string and returns the list of words of the first sentence and
the string that is left over. 

We start by implementing a function that parses the string and looks
for either a ``.'' or `` '' i.e. the end of the sentence or the end of
a word. It carries two accumulators, the characters of the last word
and the words seen so far (both in reverse order). Once it sees the
``.'' it will return a pair of the words and rest of the string. Here
is a skeleton:

\begin{verbatim}
sentence :: String -> String -> [String] -> ([String], String)

sentence (x:xs) word sofar 
  | x == '.' = (... , xs)
  | x == ' ' =  sentence xs [] (...:sofar)
  | otherwise  = sentence xs ... sofar
\end{verbatim}

Test the function and see that it works. What happens when you try a
incomplete sentence?

To fix the problem we should return a value that indicates that more
data is needed. We need to define a data type {\em Parsed} that is
either of the form {\em (Done [String] String)} or {\em More}. Make
the type an instance of {\em Show} to be able to print the
results. You could simply use the {\tt deriving} construct but you
might want to do a more customized presentation.

We can now let {\em sentence} return either depending on the outcome.

The drawback of this solution is that we have done a lot of parsing
already so we would like to continue to parse where we left off. This
can be done by including the unfinished word and the words seen so far
(both in the reverse order) in the {\em More} structure. We can then
continue at this point when we have more text to parse.

Implement the above strategy and see that you can continue to parse as
you receive more segments of the text.

Representing the state of the parser was in this case rather easy but
the state can of course be quite complex. A better idea is to return a
function that is prepared to continue parsing. This is where you need
to know your lambda expressions. The data type {\em Parsed} should
have the following definition:

\begin{verbatim}
  data Parsed = Done [String] String | More (String -> Parsed)
\end{verbatim}

In the case of an empty string the function {\em sentence} should
return a {\em More} structure that contains a single lambda
expression. The expression should take one argument, the additional
text, and call the function {\em sentence} with the correct
arguments. Change the type description of {\em Parsed} and change the
base case for {\em sentence}. If you succeed you should be able to do
the following in GHCi:

\begin{verbatim}
> let result = sentence "this is" "" []
> result
``more data needed''
> let More f = result in f " the end."
["this","is","the","end"]
""
\end{verbatim}

The advantage of this solution is that the internal representation of
the parser is well hidden to the user of the parser.









\end{document}
