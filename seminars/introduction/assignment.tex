\documentclass[a4paper,11pt]{article}


\usepackage{ifthen}
\usepackage[latin1]{inputenc}


\newcommand{\nnsection}[1]{
\section*{#1}
\addcontentsline{toc}{section}{#1}
}

\begin{document}

\begin{center}
\vspace{20pt}
\textbf{\large Erlang: getting started}\\
\vspace{10pt}
\textbf{Johan Montelius}\\
\vspace{10pt}
\today{}
\end{center}


\nnsection{Getting started}

We assume that you have Erlang up and running and an editor to write
your first Erlang programs. You don't need a full IDE, rather the
less you have to think about the better.


In this tutorial we will write input to the Erlang shall as follow:
\begin{verbatim}
> x + 2.
\end{verbatim}


\section{Simple arithmetic}

Open up a Erlang shell and try some simple arithmetic. Try these examples:

\begin{verbatim}
>6 + 2.

>6 div 2.

>7 div 2.

>7 rem 2.

>-1 rem 5.

>7 / 2.
\end{verbatim}

Some simple comparisons.

\begin{verbatim}
>3 == 3

>3 /= 4

>4 < 7
\end{verbatim}


\section{A first program}

Open up a file {\em test.erl} and create a module called {\em test}. In
this module we define a function called {\tt double} that takes one
argument and returns the double of that argument. 

\begin{verbatim}
-module(test).

double(N) -> ...
\end{verbatim}

In the Erlang shell you can load the module and try if it works.

\begin{verbatim}
>double(4).
\end{verbatim}

Now in the same module define the following functions:

\begin{itemize}
\item a function that converts from Fahrenheit to Celsius (the
  function is as follows:$$C = (F-32)/1.8$$)

\item a function that calculates the area of a rectangle give the
  length of the two sides

\item a function that calculates the area of a square, using the
  previous function

\item a function that calculates the area of a circle given the radius
\end{itemize} 

\section{Recursive definitions}

Assume that all we have is addition and subtraction but need to define
multiplications. How would you do? You will have to use recursion and
you solve it by first describing the multiplication functions by words.

{\em The product of m and n is: 0 if $m$ is equal to $0$, otherwise the
  product is $n$ plus the product of $m-1$ and $n$.}

Once you have written down the definition, the coding is simple.

\begin{verbatim}
product(M, N) -> 
   if 
     M == 0 -> ...;
     true -> ...
   end.
\end{verbatim}

There are alternative ways of writing this, we could use a {\em case
  expression}:

\begin{verbatim}
product(M, N) -> 
   case M of 
     0 -> ...;
     _ -> ...
   end.
\end{verbatim}

or simply written it as two {\em clauses}:

\begin{verbatim}
product(0, _) -> 
   0;
product(M, N) -> 
   ... .
\end{verbatim}

The last definition might be the easiest to read and is often used.

Define a function, {\tt exp/2}, that computes the exponentiation,
$x^y$. Use only the addition and subtraction and the function {\tt
  product/2}, that you defined.

\begin{verbatim}
exp(X, ...) -> 
   ...;
exp(X, Y) -> 
   ... .
\end{verbatim}

Use the built-in arithmetic functions {\em rem}, {\em div} and
multiplication {\em *} to implement a much faster exponentiation using
the following definition:

\begin{itemize}
  \item x raised to 1 is x
  \item x raised to n, if n is even, is x raised to n/2 multiplied by itself
  \item x raised to n, if n is odd, is x raised to (n-1) multiplied by x
\end{itemize}


\section{List operations}

You will do more operations on list than you have ever done before so
you might as well get use to them. These are operations that you should
know by heart.

\begin{verbatim}
>[1|[]].

>[1|[2|[]]].

>[1,2].

>[1,2] = [1|[2|[]]].

>[A,B,C] = [1,2,3]

>[H|T] = [1,2,3].

>[_, {X,Y} | _] = [{a,1},{b,2}, {c,3}, {d,4}].

>[Z] = [1,2,3].
\end{verbatim}

\subsection{Simple functions on list}

These are some simple functions that you should implement. They will
all use recursion so first try to formulate in words what the
definition should look like, then encode it using two or more clauses.

\begin{itemize}
\item nth(N, L): return the N't element of the list L
\item number(L): return the number of element in the list L
\item sum(L): return the sum of all elements in the list L, assume
  that all elements are integers
\end{itemize}


These functions take some more thinking. You should return a list as a
result of evaluating the function.

\begin{itemize}
\item duplicate(L): return a list where all elements are duplicated
\item unique(L): return a list of unique elements in the list L, that is {\tt [a,b,d]} are the unique elements in the list {\tt [a,b,a,d,a,b,b,a]}
\item reverse(L): return a list where the order of elements is reversed
\item pack(L): return a list containing lists of equal elements, {\tt
  [a,a,b,c,b,a,c,]} should return {\tt [[a,a,a], [b],[c,c]]}
\end{itemize}



\subsection{Sorting}

There are several ways to sort a list and you should know them all. We
will start with the most basic algorithm and then try some other (more or less good).

\subsection{insertion sort}

In {\em insertion sort}, you sort a list of elements by taking them
one at a time and {\em insert} them into an already sorted list. The
already sorted list will of course be empty when we start but will
when we are done contain all the elements.

Start by defining a function {\em insert(Element, List)}, that inserts
the element at the right place in the list. Think of the two mayor
cases, what to do if the list is empty and what to do if the list
contains at least one element. Assume that the elements are integers
and can be compared using the regular {\em <} operator.


Once you have {\em insert/2} working, implement the sorting function
{\em isort(List, Sorted)}; again what should you do if the list is
empty, what should you do if it contains at least one element?

Now all you have to do is provide a function {\em isort(List)}, that
calls the function {\em isort/2} using the right arguments.

\begin{verbatim}
isort(L) -> isort(L, ...).

isort(.., ...) ->
  ...;
isort(..., ...) ->
  isort(..., ...).

insert(..., ...) ->
  ...;
insert(..., ...) ->
  if 
    ... -> ...;
    true -> ...
  end.
\end{verbatim}


\subsection{merge sort}

In {\em merge sort}, you divide the list into two (as equal as
possible) list. Then you merge sort each of these lists to obtain two
sorted sub-lists. These sorted sub-lists are then {\em merged} into
one final sorted list. 

The two lists are merged by picking the smallest of the elements from
each of the lists. Since each list is sorted, one need only to look at
the first element of each list to determine which element is the
smallest.

The skeleton code below will give you an idea of what the solution
will look like.

\begin{verbatim}
msort(..) -> 
    ...;
msort(..) -> 
    ...;
msort(...) ->
   {..., ...} = msplit(..., [], []),
   merge(msort(...), msort(...)).

msplit(..., ..., ...) ->
   {..., ...};
msplit(..., ..., ...) ->
   msplit(..., ..., ...).
\end{verbatim}

\subsection{quick sort}

The {\em quick sort} algorithm sounds even quicker than merge sort but
this is not true. The idea is similar but now we ``do our sorting on
the way down''. First split the list into two parts, one containing low
elements and one containing high elements. Then sort the two lists and
when you're done append the results. 

Splitting the lists is done using the first element in the list as a
{\em pivot element}, all smaller or equal than this is added in one
list and all larger in one list. When you're appending the final
result, remember to put the pivot element in the middle.

\begin{verbatim}
qsort(...) -> ...;
qsort(...) ->
   {..., ...} = qsplit(..., ..., [], []),
   SmallSorted = ...;
   LargeSorted = ...;
   append(..., ...).

qsplit(..., ..., ..., ...) ->
  {..., ...};
qsplit(..., ..., ..., ...) ->
  qsplit(..., ..., ..., ...).
\end{verbatim}

\section{Reverse}

One interesting problem to look at is how to reverse a list. The {\em
  naive} way to do it is quite straight forward. We do it recursively
by removing the first element of the list, reversing the rest and then
appending the reversed list to a list containing only the first element.

\begin{verbatim}
nreverse([]) -> [];
nreverse([H|T]) ->
   R = nreverse(T),
   append(R, [H]).
\end{verbatim}

A smarter way to do it, is to use an {\em accumulator} and add the
first element to this accumulator. When we have added all elements in
the lists the accumulated list is the reversed list.

\begin{verbatim}
reverse(L) -> 
   reverse(L, []).

reverse([], R) ->
   R;
reverse([H|T], R) ->
   reverse(T, [H|R]).
\end{verbatim}

Ok, so what is so smart by doing this? This is your assignment, you
should do some performance analysis of these two functions and
describe what is happening. To have some data lead you in the right
direction and to back up your findings you should start by doing some
performance measurements. We have here used some library functions and
higher order programming that you might not have seen so far but 


\begin{verbatim}
bench() ->
    Ls = [16, 32, 64, 128, 256, 512],
    N = 100,
    Bench = fun(L) ->
       S = lists:seq(1,L),
       Tn = time(N, fun() -> nreverse(S) end),
       Tr = time(N, fun() -> reverse(S) end),
       io:format("length: ~10w  nrev: ~8w us    rev: ~8w us~n", [L, Tn, Tr])
       end,
    lists:foreach(Bench, Ls).

time(N, F)->
    %% time in micro seconds
    T1 = erlang:system_time(micro_seconds),
    loop(N, F),
    T2 = erlang:system_time(micro_seconds),
    (T2 -T1).

loop(N, Fun) ->
  if N == 0 -> ok; true -> Fun(), loop(N-1, Fun) end.
\end{verbatim}


\section{Binary coding}

As the next exercise, you should implement a function that takes an
integer and return its binary representation coded as a list of ones
and zeroes. The binary form of $13$ is for example {\em [1,1,0,1]}. 

There are many ways to solve this problem and you should solve it in
two different ways, describing the difference and if one is better
than the other.


\section{Fibonacci}

The Fibonacci sequence is the sequence $0,1,1,2,3,5,8,13,21,
\ldots$. The two first numbers are $0$ and $1$ and the following
numbers are calculated by adding the two previous number. To calculate
the Fibonacci value for $42$, all you have to do is find the Fibonacci
number for $40$ and $41$ and then add them together.

Write simple Fibonacci function {\em fib/1} and do some performance
measurements. Adapt the benchmark program above and run some tests.

\begin{verbatim}
fibb() ->
    Ls = [8,10,12,14,16,18,20,22,24,26,28,30,32,34,36,38,40],
    N = 10,
    Bench = fun(L) ->
       T = time(N, fun() -> fib(L) end),
       io:format("n: ~4w  fib(n) calculated in: ~8w us~n", [L, T])
       end,
    lists:foreach(Bench, Ls).
\end{verbatim}

Find an arithmetic expression that almost describes the computation
time for $fib(n)$. Can you justify this arithmetic expression by
looking at the definition of the function?  How large Fibonacci number
do you think you can compute if you start now and let your machine run
until the seminar? First make a guess, don't try to do the calculation
in your head just make a wild guess, then try to estimate how long
time that would take using your arithmetic function, would you be able
to make it? 

Calculate a Fibonacci number as high as you possibly can.

\end{document}


\grid
