\documentclass[a4paper,11pt]{article}


\usepackage[latin1]{inputenc}

\usepackage{amsmath}


\newcommand{\nnsection}[1]{
\section*{#1}
\addcontentsline{toc}{section}{#1}
}

\begin{document}

\begin{center}
\vspace{20pt}
\textbf{\large T9}\\
\vspace{10pt}
\textbf{Johan Montelius}\\
\vspace{10pt}
\today{}
\end{center}

\nnsection{Let's apply for a job}

This was an assignment that I got from one of my students. She was
applying for a job at Ericsson and she had been given an assignment
that she needed some help with. The task was quite interesting yet
very simple to describe. I can understand that they gave this as an
assignment, the solution will show how far you have reached.

The problem is to implement a technique that you probably use every
day, the quick typing on mobile phones known as T9. You're given a
list of integers that represents the keys presses and your task is to
find the possible words given a small dictionary,

For historical reasons the keys on a phone are mapped starting on $2$,
that is mapped to (a,b,c) continuing to $9$ that is mapped to
(w,x,y,z). If you're given the list {\tt [4,3,5,5,6]} you should
generate the list {\tt [``hello'']} if we have ``hello'' in the
dictionary. You should return a list of all possible words so if a
sequence can match two words both words should be in the list.

The assignment was not to give the most efficient solution but to give
an elegant solution. Let's start with a simple solution and then a 
more efficient.

\nnsection{the first solution}

Let's do an initial solution that solves the problem. Assume we have a
dictionary that simply lists all possible words.

\begin{verbatim}
dictionary :: [String]
dictionary = ["foo", "bar", "abr", "gul", "eri"]
\end{verbatim}

The idea for the first solution will look as follows. First we will
construct a {\em table} that is a list of pairs {\em (Digits, Word)}
where the digits is a list of the digits that correspond to the word. 

If we have the table of all words in the dictionary we can filter it
using one digit at a time from the sequence of digits that we are
given. When we are done we simply convert the remaining table to a
list of words.

First define a function {\em table} that produces the table. If you
follow these steps you can have it done in no time.

\begin{itemize}
\item Implement a function {\em char\_to\_digit} that takes a
  character and returns a digit.

\item Implement a function {\em encode} that takes a word and returns
  a list of digits.

\item If you map {\em encode} to the dictionary you will get a list of
  digit sequences.  

\item Use the predefined function {\em zip} and
  take the generated list and the dictionary to create the table.
\end{itemize}

When you implement these functions, try them one by one in GCHi to see
that they works as expected.

Once we have our table we can simply filter the table given the
sequence that we're given. Once we have reduces the pairs in the table
we can apply a mapping function that selects the words from the pairs
and we're done.

\nnsection{a more efficient solution}

If this second solution is more efficient actually depends on the
usage. If we only have one sequence to check it was probably already
unnecessary to construct the table in the first solution. The table
only makes sense if we can use it more than once. The second solution
will build an even more complicated table but will have much better
performance for large dictionaries.

The change we will do is to implement the table not as a list of
pairs but as a tree of degree eight, one branch for each digit (2..9). A path
from the root of the tree to a node represents a sequence of digits
and the reached node will have a set w words that corresponds to this
sequence.

We will first describe the data type {\em Node} that holds a list of
{\em Branches}, where a branch is a tuple of a digit and a node, and a
list of {\em Words}. This means that a leaf node is identified by a
empty list of branches. An empty tree consists of a node with no
branches nor words.

First implement the data types needed and the operations {\em root},
{\em insert} and {\em select}.

\begin{itemize}
\item Root will create a empty node. 
\item Insert will add a {\em Pair} (defined in the previous solution)
  to a tree.
\item Select will select a sub-tree (the node of a branch) given a digit. 
\end{itemize}

If you get these functions right you should be able to construct a tree
from the table. You should also be able to implement the main function
of T9 that should take a sequence of digits and first selects the
sub-tree and then returns the corresponding words.

\nnsection{Generic tree patterns}

One can of course implement a set of generic tree algorithms that
traverses, search or adds information to a generic tree. This is
however hardly worth the trouble since trees are not as uniform as
lists. In our case we had branches represented a list of pars where
most trees would have a left and a right branch. We would have to
parametrize the algorithm to use given functions for navigating down
the branches and accessing node data. 



\end{document}
