\documentclass[a4paper,11pt]{article}


\usepackage[latin1]{inputenc}

\usepackage{amsmath}

\newcommand{\nnsection}[1]{
\section*{#1}
\addcontentsline{toc}{section}{#1}
}

\begin{document}

\begin{center}
\vspace{20pt}
\textbf{\large Recursion}\\
\vspace{10pt}
\textbf{Johan Montelius}\\
\vspace{10pt}
\today{}
\end{center}

\nnsection{Simple arithmetic functions}

Implement the following arithmetic functions and save them in a file
{\tt recursion.erl}. Start with simple recursive implementations that
are not necessarily tail-recursive. In order not to create any
conflicts with builtin function, you will have to call your
functions something different i.e. prod instead of product etc.

When you implement them: write a small documentation, write the type
definition and the definition, then try them using GHCi. For example:

\begin{verbatim}
-module(recursion).

%% product of n and m : 
%%   if n is 0 
%%          then ....
%%     otherwise 
%%          the result ...


prod(.., ..) -> 
    case ...  of
       ... ->  ...;
       ... ->  ...
    end.
\end{verbatim}


\begin {itemize}
\item prod: complete the above implementation of product
\item prod: change the above definition so that it also works for
  negative numbers
\item power: implement the power function for non-negative exponents
\item qpower: use the two functions $div$ and $rem$ to implement a
  faster version of the power function by dividing the exponent by two
  in each recursive step (what must be done if the exponent is odd?)
\end{itemize}

\nnsection{Binary}

Implement a function {\tt binary/1} that takes an integer and produces its binary
encoding, represented ad a list of ones and zeros. Your firts solution
should look something like this:

\begin{verbatim}
binary(0) ->
    [];
binary(...) ->
    binary(...) ++ [...].
\end{verbatim}

Note that the {\tt ++ } operator is just a short form of writing {\tt
  append/2}. What is the deficiency of this implementation. What is
the runtime complexity?

Change your implementation so that it has a {\em O(n)} komplexity. The
solution will look something like this:

\begin{verbatim}
binary2(I) ->
    binary(I, ...).

binary2(0, Sofar) ->
    lists:reverse(Sofar);
binary2(..., ....) ->
    binary2(..., ...).
\end{verbatim}

We're using the library function {\tt lists:reverse/1} to reverse the
acumulated value.




\nnsection{Fibonacci}

Try some more complex functions, for example the Fibonacci function:

\[
  fib(n) = \left\{
  \begin{array}{l l}
    0 & \quad \text{if $n$ is 0}\\
    1 & \quad \text{if $n$ is 1}\\
    fib(n-1)+fib(n-2) & \quad \text{otherwise}\\
  \end{array} \right.
\]

The straight forward implementation is a bit inefficient, when
evaluating fib(n-1) in the recursion we of course learn fib(n-2) as a
by product but this is not taken advantage of. Can you change the
implementation so that it only computes each Fibonacci number once? 

\nnsection{Ackermann}

You can also give the Ackermann function a try:

\[
  ackerman(m,n) = \left\{
  \begin{array}{l l}
    n+1 & \quad \text{if $m=0$}\\
    ackerman(m-1,1) & \quad \text{if $m>0$ and $n=0$}\\
    ackerman(m-1,ackerman(m,n-1)) & \quad \text{otherwise}\\
  \end{array} \right.
\]

This looks like an innocent little function but don't try too high numbers :-) 

\nnsection{list processing}

Open up new module and implement the following list processing
functions (again we will have to call them something not that obvious
in order not to create a conflict with the functions in Prelude):

\begin{itemize}
\item drp: drop the first $n$ elements of a list
\item tak: take the first $n$ elements of a list
\item append: append two lists
\item rev: reverse a list
\item palindrome: return $True$ if a list is a palindrome.
\end{itemize}

\nnsection{sorting}

The final assignment will be to implement the {\em merge sort}
algorithm. Open up a new file {\tt Sort.hs} and start with a new module.

\begin{verbatim}
module Sort
       where
\end{verbatim}

To implement the algorithm we need some additional tools. The algorithm
is straight forward i.e. split a list in two, sort the two parts and
then merge them together, but the tricky issue is how to write a
function that returns two list. The simplest way to do this is to
return a tuple that consist of two elements. The syntax for tuples is
{\tt (a,b)} and the split function has the following type:

\begin{verbatim}
split :: Ord a => [a] -> ([a], [a])
\end{verbatim}

To implement {\tt split} we might want to use a help function, a
version of split that takes three arguments: the rest of the list that
we should split, half of the elements that we have seen and the other
half. Call this function {\tt distribute} and then define {\tt split}
as follows:

\begin{verbatim}
split xs = distribute xs [] []
\end{verbatim}

The next function we need is the {\tt merge} function. It should take
two sorted lists and merge them into one sorted lists. If one of the
two lists is empty he other list is of course the result, if not we
must compare the two first elements of the two lists. 

\begin{verbatim}
merge :: Ord a => [a] -> [a] -> [a]
merge [] ys = ...
merge xs [] = ...
merge (x:xs) (y:ys) = ..... | ....
merge (x:xs) (y:ys) = ..... | ....
\end{verbatim}

Time to wrap it up, we now need a construct that lets us first split
the list and then work on the two sub lists. This is done using a {\tt
  let in} expression. The expression allows us to perform one or more
function calls to evaluate results that we need in our main expression. 

\begin{verbatim}
msort :: Ord a => [a] -> [a]
msort [] = ...
msort [x] = ...
msort xs = 
   let 
     (as, bs) = split xs
   in
     merge ... ...
\end{verbatim}

I this definition we use the let expression to get direct access to
the two list that {\tt split} returns. The {\tt (as,bs)} on the right
hand side works as a pattern that the result is matched with. The
variables {\tt as} and {\tt bs} will as a result be bound to the two
lists.

If you have time you can try to implement quick sort. You might find
that an {\tt if} expression comes handy:

\begin{verbatim}
   if x < p then (x:lo, hi) else (lo, x:hi)
\end{verbatim}


\end{document}
