\input{../include/preamble}


\title[ID1019 OTP]{OTP}


\author{Johan Montelius}
\institute{KTH}
\date{\semester}

\begin{document}

\begin{frame}
\titlepage
\end{frame}

\begin{frame}{a framework}

OTP, Open Telecom Platform, the system that we have used:

\pause\vspace{20pt}
\begin{itemize}
\item compiler and run-time system
\item set of libraries
\item generic templates to implement servers  
\item tools and services: database servers, parsers, debugger etc
\end{itemize}

\end{frame}


\begin{frame}[fragile]{a server}

\begin{verbatim}
gen_start(Init) ->
 spawn(fun() -> init(Init) end).

init(Init) -> 
  :
 server(State).

server(State) ->
 receive
   {request, Msg, From}->
          :
      From ! Reply,
      server(Updated);
    :
 end.           
\end{verbatim}

\end{frame}

\begin{frame}{generic and specific}

What is generic, same for any server, and what is specific.

\pause\vspace{20pt}

An important factor in any systems development is to work with layers
of abstraction. 

\vspace{20pt}
Separate the generic behavior of a set of services in one abstraction, and
then use this abstraction when defining the actual services.

\end{frame}

\begin{frame}[fragile]{abstractions}

\begin{verbatim}
sum([]) -> 0;
sum([H|T]) -> H + sum(T).
\end{verbatim}

\pause\vspace{20pt}

\begin{verbatim}
sum(L) -> foldr(fun(H,A) -> H + A end, 0, L).
\end{verbatim}

\end{frame}

\begin{frame}[fragile]{a server}

\begin{verbatim}
gen_start(Init) ->
 spawn(fun() -> init(Init) end).

init(Init) -> 
  :
 server(State).

server(State) ->
 receive
   {call, Msg, From}->
          :
      From ! Reply,
      server(Updated);
    :
 end.           
\end{verbatim}

\end{frame}

\begin{frame}[fragile]{a generic server}

\begin{verbatim}
start(Init) -> spawn(fun() -> gen_init(Init) end).

gen_init(Init) ->
 {ok, State} = init(Init),
 server(State).

server(State) ->
 receive
   {call, Msg, From}->
      {reply, Reply, Updated} = handle_call(Msg, State),
      From ! Reply,
      server(Updated);
    :
 end.           
\end{verbatim}

\end{frame}


\begin{frame}[fragile]{a generic server}

\begin{verbatim}
-module(gen_server).

start(Module, Init) -> 
   spawn(fun() -> init(Module, Init) end).

init(Module, Init) ->
 {ok, State} = apply(Module, init, [Init]),
 server(Module, State).
\end{verbatim}


\end{frame}

\begin{frame}[fragile]{a generic server}

\begin{verbatim}
server(Module, State) ->
 receive
   {call, Msg, From}->
      {reply, Reply, Updated} =  apply(Module, handle_call, [Msg, State]),
      From ! Reply,
      server(Module, Updated);

    :

 end.           
\end{verbatim}

\end{frame}

\begin{frame}[fragile]{the server}
\begin{verbatim}
-module(account).

start(Init) ->
  gen_server:start(account, Init).

init(State) ->
  {ok, State}.

hande_call(saldo, _From, State} ->
  {reply, {saldo, State}, State}.

\end{verbatim}

\end{frame}

\begin{frame}{gen\_server}

The {\tt gen\_server} definition:

\begin{itemize}
\item {\tt init(Args)}: return {\tt \{ok, State\}} or {\tt \{stop, Reason\}}.
\pause
\item {\tt handle\_call(Request, From, State)} : return 
 \begin{itemize} 
  \item {\tt \{reply, Reply, New\_state\}} : Reply is sent to the requester
\pause
  \item {\tt \{noreply, New\_state\}} : no reply is returned
\pause
  \item {\tt \{stop, Reason, New\_State\}} :  terminate/2 is called
 \end{itemize}
\pause
\item {\tt handle\_cast(Request, State)} : similar but no replies
\pause
\item {\tt terminate(Reason, State)} : opposite to init/1
\end{itemize}

\pause\vspace{20pt}
{\em there are more}

\end{frame}

\begin{frame}{the gen\_server api}

The {\tt gen\_server} interface:

\pause\vspace{20pt}

\begin{itemize}
\item {\tt gen\_server:start(Module, Args, Options)} : starts the server defined in Module
\pause
\item {\tt gen\_server:call(Server, Request)} : returns a reply from the server
\pause
\item {\tt gen\_server:cast(Server, Request)} : sends the request, no reply
\end{itemize}

\pause\vspace{20pt}
{\em there are more}

\end{frame}



\begin{frame}[fragile]{the server}
\begin{verbatim}
-module(account).

start(Init) ->
  gen_server:start(account, Init).

saldo(Account) ->
  gen_server:call(Account, saldo).

init(State) ->
  {ok, State}.

hande_call(saldo, _From, State} ->
  {reply, {saldo, State}, State}.

\end{verbatim}

\end{frame}

\begin{frame}[fragile]{gen\_server}

The {\em gen\_server} defines a server with a request-reply pattern. 

\pause\vspace{20pt}
There is no implicit deferral of messages!

\pause\vspace{20pt}
All messages must be handled by the {\tt handle\_call} and {\tt handle\_cast} functions.

\pause\vspace{20pt}
A compiler directive in the module is used to have the compiler check that all necessary functions are included:

\begin{verbatim}
-behaviour(gen_server).
\end{verbatim}

\end{frame}



\begin{frame}[fragile]{gen\_fsm}

\pause\vspace{20pt}
There is a sister behavior {\tt gen\_fsm}, that tries to capture the
generic behavior of a finite state machine.


\end{frame}


\begin{frame}[fragile]{event handlers}

What is an ``event''?

\pause\vspace{20pt} 
Send all events of a particular type to an {\em event manager process}. 

\pause\vspace{20pt} 
Register {\em handlers modules} wit the manager process.

\pause\vspace{20pt} 
Each time an event is received by the manager, all handlers will be called.

\end{frame}

\begin{frame}[fragile]{gen\_event}

\begin{verbatim}
-module(error_handler).
-export([init/1, handle_event/2, terminate/2]).

init(Name) ->
    {ok, []}.

handle_event({error, Msg}, []) ->
   io:format("error: ~w~n", [Msg]),
   {ok, []};
handle_event(_, []) ->
   {ok, []};

   :
   :

terminate(_Args) ->
    ok.
\end{verbatim}

\end{frame}


\begin{frame}[fragile]{gen\_event}

\begin{verbatim}
-module(test).

start() ->
   Manager = gen_event:start({local, tracer}),
   gen_event:add_handler(tracer, error_handler, []),
   server().
\end{verbatim}



\begin{verbatim}
server() ->
  receive 
    {expected, Msg} ->
        handle(Msg),
        server();
    Error ->
        gen_event:notify(tracer, {error, Error}),
        server()
  end.
\end{verbatim}

\end{frame}


\begin {frame}[fragile]{supervision of processes}

\begin{verbatim}
start(Init) ->
   process_flag(trap_exit, true),
   Server = gen_server:start_link(my_server, Init, []),
   supervise(Server, Init).
\end{verbatim}

\pause\vspace{20pt}
\begin{verbatim}
supervise(Server, Init) ->
  receive
    {'EXIT',Server, _Reason} ->
       Restarted = gen_server:start_link(my_server, Init, []),
       supervise(Restarted, Init);
    terminate ->
       gen_server:cast(Server,stop_what_your_doing)
 end.
\end{verbatim}

\end{frame}

\begin{frame}{the generic supervisor}

\begin{itemize}
\item {\tt init(Args)} : returns {\tt \{ok, Modes, Specs\}} that describes which servers should be started and how the supervisor should operate
\end{itemize}

\pause\vspace{20pt}

\begin{itemize}
 \item {\tt Modes} : {\tt \{Strategy, Allowed, Time\}}
 \pause
 \begin{itemize}
  \item {\tt one\_for\_one} : if a server dies, restart it
  \pause
  \item {\tt all\_for\_one} : terminate all servers and restart them 
  \pause
  \item {\tt rest\_for\_one} : terminate the ones after the server that died and then restart them
  \pause
 \end{itemize}
\end{itemize}

\end{frame}

\begin{frame}{the generic supervisor}

\pause\vspace{20pt}

\begin{itemize}
 \item {\tt Specs} : {\tt \{Id, \{Module, Name, Args\}, Restart, Shutdown, Type, Modules\}}
 \pause
 \begin{itemize}
  \item {\tt Id} : a name local to the supervisor
  \pause
  \item {\tt \{Module, Name, Args\}} : description of the server to start
  \pause
  \item {\tt Restart} : {\tt transient} (do not restart), {\tt temporary} (restart if exception) or {\tt permanent} (always restart)
  \pause
  \item {\tt Shutdown} : for how long is the process allowed to do clean-up when terminated
  \pause
  \item {\tt Type} : a {\tt worker} or {\tt supervisor} 
  \pause
  \item {\tt Modules} : used when code base is updated 
 \end{itemize}
\end{itemize}

\end{frame}

\begin{frame}{a supervisor tree}

\begin{tikzpicture}[>=stealth', node distance=2.5cm, semithick, auto]
    \node[state]        (s1)                              {supervisor};
    \node[state]        (s2)     [below right of=s1]      {supervisor};
    \node[state]        (w1)     [below left of=s1]       {worker};
    \node[state]        (w2)     [left of=w1]             {worker};
    \node[state]        (w3)     [left of=w2]             {worker};


    \node[state]        (w4)     [below right of=s2]       {worker};
    \node[state]        (w5)     [left of=w4]             {worker};

    \path[->]   (s1)      edge       []    node      {}    (s2);
    \path[->]   (s1)      edge       []    node      {}    (w1);
    \path[->]   (s1)      edge       []    node      {}    (w2);
    \path[->]   (s1)      edge       []    node      {}    (w3);
    \path[->]   (s2)      edge       []    node      {}    (w4);
    \path[->]   (s2)      edge       []    node      {}    (w5);
\end{tikzpicture}

\end{frame}

\begin{frame}

An OTP server architecture becomes quite complicated. 

\pause\vspace{20pt}
It is over-kill for smaller systems or when you're all alone.

\pause\vspace{20pt}
It is invaluable for larger systems or if you are more than one programmer.

\pause\vspace{20pt}
Try first to implement a supervisor tree from scratch, then you will
appreciate (and understand) the power of the OTP framework.

\pause\vspace{20pt}
If you start by learning OTP without having done it yourself, you will
most likely not understand how it works or why it is needed.


\end{frame}

\begin{frame}{other OTP frameworks}

\begin{itemize}
\item {\tt application}: handle a set of modules, and possibly a supervisor tree,  as one component
\pause
\item {\tt release} : a set of Erlang {\tt applications} and additional information that should be treated as a release
\pause
\item {\tt release handling} : updating a running system to the next release
\pause
\item {\tt rebar} : (not a proper part of OTP), the Erlang framework to automatically build applications and releases 
\end{itemize}
\end{frame}

\begin{frame}{Summary}

The OTP framework is invaluable when developing larger applications. 

\pause\vspace{20pt}

You should not start using it before you know how to program in Erlang.



\end{frame}



\end{document}


